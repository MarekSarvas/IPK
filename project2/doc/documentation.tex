\documentclass{article}
\usepackage{url}
\usepackage{array}
\newcolumntype{L}{>{\centering\arraybackslash}m{4cm}}
\usepackage[english]{babel}
\usepackage[utf8]{inputenc}
\usepackage[unicode]{hyperref}
\usepackage{graphicx}
\usepackage{textcomp}
\usepackage[T1]{fontenc}
\usepackage[left=2cm, text={17cm, 24cm}, top=2cm]{geometry}
\usepackage[table,xcdraw]{xcolor}
\usepackage{caption}
\usepackage{color}
\usepackage{hyperref}
\hypersetup{
    colorlinks=true, % make the links colored
    linkcolor=blue, % color TOC links in blue
    urlcolor=red, % color URLs in red
    linktoc=all % 'all' will create links for everything in the TOC
}

\begin{document}

%%%%%%%%%%%%%%%%%TITULKA%%%%%%%%%%%%%%%%%% 

	\begin{titlepage}
		\begin{center}
			\textsc{\Huge Vysoké Učení Technické v Brně} \\[0.7cm]
			{\Huge Fakulta informačních technologií}
			\center\includegraphics[width=0.5\linewidth]{./logo.png}

			\vspace{5cm}

			\textbf{{\Huge Documentation for IPK - Project 2}}\\[0.4cm]
			\textbf{{\LARGE Packet sniffer Implementation}}\\[0.4cm]
	
			
		\end{center}
		\vfill

		\begin{flushleft}
			\begin{Large}
				
				Marek Sarvaš\hspace{37px}(xsarva00)\hspace{19px} 
			\hfill
			Brno, 11.12.2019
			\end{Large}
		\end{flushleft}

	\end{titlepage}
%%%%%%%%%%%%%%%%%TITULKA%%%%%%%%%%%%%%%%%% 

%%%%%%%%%%%%%%%%%OBSAH%%%%%%%%%%%%%%%%%% 

	\tableofcontents
	\newpage
%%%%%%%%%%%%%%%%%OBSAH%%%%%%%%%%%%%%%%%% 


	\section{About}
	\large{ABOUT}
	\newpage
	\section{Implementation}

Sniffer is implemented in one file ipk-sniffer, whole program is divided into few functions and main function. \newline
MAIN
First part of main function is for parsing given arguments using check\textunderscore args function.If interface was not given as argument all interfaces are printed in loop.
	If program got interface ( or other optional arguments such as port, tcp, etc.) 
	Firstly opens given interface for sniffing using pcap \textunderscore open \textunderscore live, on success compile given filter composed from given program arguments - tcp, udp, port. If compiled successfuly filter is applied on interface handler.
	For actual sniffing pcap \textunderscore loop function is used with arguments interface handler, number of packets to be sniffed ( stored int argument structure), callback function ( documented below). There is no time limit in which packet has to be sniffed, because if user wants to sniffed eg. 2 packets program will run until 2 packets are sniffed.  After wanted number of packets is sniffed programme closes interface handler and frees allocated memory. Otherwise if interface where tcp/udp packets could be sniffed were given sniffer will run infinitely until interuption ( eg.: with CTRL+C ).\newline
CHECK \textunderscore ARGS
Verify arguments given to program using getopt, in case of -p ( port filter ) convert port number into integer and checks its correct value which has to be between 0 and 65535. \newline
CREATE \textunderscore FILTER
Creates a string filter using tcp, udp, port number from values given as programme arguments eg.: "port 80", "tcp", "udp port 5353" etc. Because programme is sniffing only tcp or udp packets filter "tcp udp" is same as none.\newline

CALLBACK\newline
Function passed into pcaploop called for every packet sniffed. Is responsible for parsing packet to get necessary information such as: time, protocol of packet, source and destination ports and ip addresses; resolving ip addresses into names and printing these information and whole packet on standard output.



	\section{Testing}
	testing
	

	
	

	
\end{document}
