\documentclass{article}
\usepackage{url}
\usepackage{array}
\newcolumntype{L}{>{\centering\arraybackslash}m{4cm}}
\usepackage[english]{babel}
\usepackage[utf8]{inputenc}
\usepackage[unicode]{hyperref}
\usepackage{graphicx}
\graphicspath{ {../test_img/} }
\usepackage{textcomp}
\usepackage[T1]{fontenc}
\usepackage[left=2cm, text={17cm, 24cm}, top=2cm]{geometry}
\usepackage[table,xcdraw]{xcolor}
\usepackage{caption}
\usepackage{color}
\usepackage{hyperref}
\hypersetup{
    colorlinks=true, % make the links colored
    linkcolor=blue, % color TOC links in blue
    urlcolor=red, % color URLs in red
    linktoc=all % 'all' will create links for everything in the TOC
}
\usepackage[numbib]{tocbibind}

\begin{document}

%%%%%%%%%%%%%%%%%%%%%%TITLE%%%%%%%%%%%%%%%%%%%%%%%% 

	\begin{titlepage}
		\begin{center}
			\textsc{\Huge Vysoké Učení Technické v Brně} \\[0.7cm]
			{\Huge Fakulta informačních technologií}
			\center\includegraphics[width=0.5\linewidth]{./logo.png}

			\vspace{5cm}

			\textbf{{\Huge Documentation for IPK - Project 2}}\\[0.4cm]
			\textbf{{\LARGE Packet sniffer Implementation}}\\[0.4cm]
	
			
		\end{center}
		\vfill

		\begin{flushleft}
			\begin{Large}
				
				Marek Sarvaš\hspace{37px}(xsarva00)\hspace{19px} 
			\hfill
			Brno, 03.05.2020
			\end{Large}
		\end{flushleft}

	\end{titlepage}
%%%%%%%%%%%%%%%%%%%%%%TITLE%%%%%%%%%%%%%%%%%%%%%%%% 

%%%%%%%%%%%%%%%%%TABLE OF CONTENT%%%%%%%%%%%%%%%%%% 

	\tableofcontents
	\newpage
%%%%%%%%%%%%%%%%%TABLE OF CONTENT%%%%%%%%%%%%%%%%%% 
\section{Preface}
Documentation for packet sniffer implemented in C++ language with libraries for manipulating with packets, for neccessary header structures such as ethernet header, ip header tcp header etc., namely \textbf{pcap.h, netinet/ip.h, netinet/ip6.h, netinet/tcp.h} etc.. Programme is sniffing packets using IPv4/IPv6 and UDP/TCP protocol on various ports.  
\section{Theory}
\subsection*{Transport layer}
Is 4th layer which transports application-layer messages between application endpoints using TCP and UDP protocols( in the internet ). It breaks application messages into segments i.e. packets and sends them into internet layer where the recieving side reassembles them and passes to application layer.
\subsection*{Packet}
Packet is an unit that carries data over network, it represents the smallest amount of data that can be transferred over a network at once. It contains control information( source destination addresses, error detection and correction etc. ) and the data it is carrying. User data are encapsulated between header and trailer where control information are carried.
\subsection*{TCP}
It is connection-oriented protocol in which the connection between client and server is established before any data is sent. TCP uses three way handshake for better error detection and reliability but adds on latency. The minimum size of header is 20 bytes and maximum 60 bytes where main segments used in this project  where \textit{source port} and \textit{destination port}, each of them takes 16 bits.
\subsection*{UDP}
UDP is another transport layer protocol but is unreliable and connectionless unlike TCP. It does not use three way handshake because there is no need to establish connection before data transfer. Using UDP performance is heigher it does not check for errors, drops delayed packets and hase better latency than TCP. It is highly used in pc gaming or video communication. UDP header has fixed length to 8 bytes and contains neccessary information for this project such as \textit{source port} and \textit{destination port} with same 16 bits length as TCP.
\newpage


\section{Implementation}
	Sniffer is implemented in one file ipk-sniffer, whole program is divided into few 				functions and main function. Compilation and how to run the sniffer is 

\subsection{Main()}
First part of main function is for parsing given arguments using check\textunderscore args function.If interface was not given as argument all interfaces are printed in loop.\newline
	If program got interface ( or other optional arguments such as port, tcp, etc.), 
	opens given interface for sniffing using pcap\textunderscore open\textunderscore live, on success compile given filter composed from given program arguments - tcp, udp, port. If compiled successfuly filter is applied on interface handler.\newline
	For actual sniffing pcap \textunderscore loop function is used with arguments interface handler, number of packets to be sniffed ( stored int argument structure), callback function ( documented below). There is no time limit in which packet has to be sniffed, because if user wants to sniffed eg. 2 packets program will run until 2 packets are sniffed.  After wanted number of packets is sniffed programme closes interface handler and frees allocated memory.\newline
	 Otherwise if interface, where tcp/udp packets could not be sniffed was given, sniffer will run infinitely until interuption ( eg.: with CTRL+C ).\newline
	
\subsection{Create\textunderscore filter}
Creates a string filter using tcp, udp, port number from values given as programme arguments. Because programme is sniffing only tcp or udp packets default filter will be set to \textbf{"(proto tcp) or (proto udp)"}, this filter is set either if program is run with "--tcp" and "--udp" arguments or without them. "Proto" in filter means that only IPv4 and IPv6 packets will be sniffed and default tcp/udp filter means only tcp and udp packets will be sniffed. Port number is added into filter with \textbf{"and"} e.g.: \textbf{"(proto tcp) and (port 443)"}.

\subsection{Callback fucntion}
Function passed into pcap\textunderscore loop and is called for every packet sniffed. Is responsible for parsing packet to get necessary information such as: time, protocol of packet, source and destination ports and ip addresses, resolving ip addresses into names and printing these information and whole packet on standard output.
\begin{itemize}
 	\item \textbf{Ethernet type} \newline
 	
 	Firstly gets ethernet header from packet data and chooses if sniffed packet uses IPv4 or IPv6. According to IP gets IPv4/IPv6 header in which are neccessary information for future use such as IP source address and IP destination address, protocol( TCP/UDP) and length of IP header. For IP addresses \textbf{inet\textunderscore ntop} function is used with \textbf{AF\textunderscore INET} argument for IPv4 and \textbf{AF\textunderscore INET6} for IPv6.
 	
	\item \textbf{Source and destination names using cache} \newline
	
	If the IP address is not in DNS cache, it is resolved to FQDN using \textbf{getaddrinfo} function to get rid of IPv4-IPv6 dependencies. Needed information are stored in \textbf{addrinfo} structure. This structure is next used in \textbf{getnameinfo} function to get FQDN from \textbf{ai\textunderscore addr} variable. \textbf{Getnameinfo} function is used with textbf{NI\textunderscore NAMEREQD} flags which returns error if IP address cannot be resolved, this is causing small memory leak according to \textit{valgrind} even though everything is freed same as when IP address is resolved into FQDN. In both cases(IP address can or cannot be resolved) IP address is stored in DNS cache. If IP is already in cache none of above functions is called and FQDN is got from there.
	DNS cache is implemented as unordered map whre \textit{key} is IP address and \textit{value} is either resolved FQDN or same IP address. Cache is implemented because \textbf{getaddrinfo} and \textbf{getnameinfo} are producing another packets when called, then sniffer catches these new generated packets this can cause loops and influence output. 
	\newpage
	
	\item \textbf{Print packet} \newline
	
	Firstly is printed time when packet was sniffed which is got from packet header and computed to hh:mm:ss.ffff format using \textbf{convert\textunderscore time} function. Then IPv4/IPv6 source/destination IP addresses with corresponding ports. For port numbers \textbf{ntohs} function is used. Next is printed whole packet in bytes in hexadecimal and ascii format, 16  bytes per row, every row starting with hexadecimal number which represents how many bytes were printed before current row. Format of packet ouput is same as packet representation in \textbf{Wireshark}.

\end{itemize}




\section{Testing}
For testing purpose \textbf{Wireshark} application was used. Sniffer was tested only manually, filters number of packets etc., no automated testing was involved, for different configuration. Output of implemented ipk-sniffer was compared with the same packet in Wireshark for same time, source and destination ports and addresses length and data of packet.

\subsection{Basic filter testing}
This involved testing differen configuration of filter i.e. combination of \textbf{tcp, udp} and \textbf{port number}. Also manually added in code filter for IPv6 but IPv6 address did not occure that often in ethernet traffic so that was a bit time consuming.\newline
\begin{itemize}
  \item \textbf{TCP protocol and port 443 filter}\newline
  \includegraphics[scale=0.3]{basic_filter}
 \item \textbf{Without filter arguments} - TCP and UDP are default\newline
  \includegraphics[scale=0.3]{default_filter}
\end{itemize}

\newpage
\subsection{Name resolving with cache}
When resolving FQDN from ip address using \textit{getaddrinfo()} and \textit{getnameinfo()} additional packets are sent. This fact with combination of more packet sniffing e.g.: argument -n 10 can result into repeatedly sniffing only these packet for name resolving.\newline
Examples are run with \textit{\textbf{-n 20}} configuration and tested in \textbf{Wireshark} application.

\begin{itemize}
  \item \textbf{without cache} excessive packets are sent\newline
  \includegraphics[scale=0.4]{wireshark_nocache}
 \item \textbf{using cache} number of these packets is reduced to minimum\newline
  \includegraphics[scale=0.4]{wireshark_cache}
\end{itemize}
\newpage


\begin{thebibliography}{9}

\bibitem{Computer Networking}
James F. Kurose, Keith W. Ross. 
\textit{Computer networking : a top-down approach}. 
-6th edition

\bibitem{Protocol Numbers} 
Protocol Numbers,
\\\texttt{https://www.iana.org/assignments/protocol-numbers/protocol-numbers.xhtml}

\bibitem{Linux man} 
Linux manual,
\\\texttt{https://linux.die.net/man/}

\bibitem{Pcapfunctions} 
WinPcap Unix-compatible Functions,
\\\texttt{https://www.winpcap.org/docs/docs\textunderscore 40\textunderscore 2/html/group\textunderscore \textunderscore wpcapfunc.html}

\bibitem{Linux man} 
LibPcap,
\\\texttt{https://www.tcpdump.org/pcap.html}

\bibitem{Pcaploop} 
tcpdump pcap\textunderscore loop,
\\\texttt{https://www.tcpdump.org/manpages/pcap\textunderscore loop.3pcap.html}

\bibitem{Filter} 
Packet filter,
\\\texttt{https://www.tcpdump.org/manpages/pcap-filter.7.html}

\end{thebibliography}

	
	
\end{document}
